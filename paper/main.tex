\documentclass{scrartcl}
\usepackage[utf8]{inputenc}
\usepackage[T1]{fontenc}
\usepackage[ngerman]{babel}
\usepackage{amsmath}
\usepackage{lmodern} % bessere Schrift für PDFs
\usepackage{graphicx}
\usepackage{listings}
\usepackage{color}
\usepackage{comment}
\usepackage{float}

%\usepackage{lipsum} % Nur zum Auffüllen von Text

% add subdirectory for images
\graphicspath{{figures/}}


% Hurenkinder und Schusterjungen verbieten
% wtf???????
\clubpenalty=10000
\widowpenalty=10000

% neue Absatz wird mit Leerzeile begonnen:
\parindent 0pt
\parskip 12pt

\title{Multilabel Klassifikation mit Support Vector Machines und Latent Dirichlet Allocation als Featurereduktion}
\author{AutorInnen}

\date{\today}


\begin{document}

\maketitle

\begin{abstract}
    %Multilabel-Textklassifikation mit Support Vector Machines und Latent Dirichlet Allocation zur Dimensionsreduktion.
    Diese Arbeit beschreibt die Ergebnisse einer Multilabel-Textklassifikation von mathematischen Publikationen aufgrund deren Titeln und Abstracts.
    Es wurden zwei verschiedene Ansätze zur Multilabel-Klassifikation umgesetzt und miteinander verglichen.
    Die Publikationen wurden einmal direkt als Vektor-Space-Modell dargestellt und mit Hilfe von Support Vector Machines (SVM) klassifiziert.
    Der zweite Ansatz implementiert als Zwischenschritt eine Dimensionsreduktion mit Latent Dirichlet Allocation (LDA).
    Die durchgeführten Experimente zeigen, dass der direkte Ansatz mit SVM bessere Ergebnisse liefert als das Verfahren, das LDA benutzt.
\end{abstract}

\tableofcontents
\newpage

\section{Einleitung}

\begin{comment}
Einführung und Einbettung des Problems
Was ist die konkrete Problemstellung / Fragestellung der Arbeit
Literature Review
\end{comment}

Die Klassifikation von Texten ist in den letzten Jahren, mit der steigenden Anzahl an Informationen und Entstehung von Textsammlungen, zu einem wichtigen Werkzeug für Organisation und Handhabung geworden.
Viele praktische Anwendungen nutzen Textklassifikation als Verfahren, um die Benutzung von Texten zu vereinfachen und zugänglicher zu machen.
Beispiele dafür sind Bibliothekssysteme, Empfehlungssysteme, Spamerkennung, Erkennung der Sprache in Texten oder der Themen in Nachrichten.

Traditionell wird bei Textklassifikation jedem Dokument $d$ exakt ein Label $l$ aus einer Menge von Labels $L$ zugewiesen.
Wenn $|L| = 2$, dann spricht man von einem binären Klassifikationsproblem, wenn $|L| > 2$ -- von einem Multiclass-Problem.
Bei Multilabel-Textklassifikation werden Dokumenten eine Menge von Labels $Y \in L$ zugeordnet.
Die Multiclass-Klassifikation hat auch ihren Ursprung in der Textklassifikation, da dort häufig die Anforderung besteht, Dokumente in verschiedene Gruppen (mehr als 2) einzuteilen (ein Beispiel hierfür wäre die Einsortierung von Nachrichten in verschiedene Kategorien).
Das Gleiche gilt auch für die Multilabel-Klassifikation, da es oftmals sinnvoll ist, ein Dokument gleichzeitig mehreren Kategorien zuzuordnen (die selbe Nachricht könnte z.B. gleichzeitig die Labels ``Politik'', ``Wirtschaft'' und ``International'' erhalten).

In dieser Arbeit betrachten wir die Auswirkungen von der \emph{Latent Dirichlet Allocation} (LDA) zur Vorverarbeitung für die Multilabel-Klassifikation mit linearen \emph{Support Vector Machines} (SVM).
Es wird überprüft, inwiefern eine Dimensionsreduktion des Featureraums mit Hilfe der LDA eine optimalere Lösung des Multilabel-Klassifikation-Problems liefert als die direkte Anwendung linearer SVM auf den selben Daten.

\subsection{Datensatz}
\label{sub:datensatz}
%woher kommt der Datensatz?
Der untersuchte Datensatz besteht aus $1.1$ Millionen mathematischen Publikationen mit Titel, Abstract, Klassen und Erscheinungsjahr.
Der schematische Aufbau des Datensatzes wird in Tabelle \ref{tab:data} dargestellt.

%[Doc:] [ID] [LABELS] [YEAR] [TITLE] [ABSTRACT]
\begin{table}[h]
    %\rowcolors[]{1}{blue!20}{blue!10}
    \begin{tabular}{cccll}
        \tiny\textbf{ID} &\tiny \textbf{CLASSES} &\tiny \textbf{YEAR} &\tiny \textbf{TITLE} & \tiny \textbf{ABSTRACT} \\
        \hline
        \tiny 1000000796 &\tiny EW & \tiny 2000 & \tiny Dynamo: A transparent dynamic \dots  & \tiny We describe the design \dots \\
        \tiny 1000000815 &\tiny EW & \tiny 2000 & \tiny Unification-based pointer \dots  & \tiny This paper describes \dots \\
        \multicolumn{5}{c}{\dots} \\
        \tiny 1000003814 &\tiny EV,PE & \tiny 1995 & \tiny A GENETIC APPROACH TO \dots  & \tiny The quadratic assignment \dots \\
        \tiny 1000004333 & \tiny PQ,EX &\tiny 1995 &\tiny DYNAMIC EUCLIDEAN MINIMUM \dots & \tiny We maintain the minimum \dots \\
    \end{tabular}
    \caption{Auszug aus dem Datensatz}
    \label{tab:data}
\end{table}


Die Publikationen werden in 14 verschiedene Klassen eingeteilt.
Für $75.6 \%$ der Dokumente wurde nur eine Klasse vergeben, für $24.3 \%$ -- zwei Klassen und für $0.1 \%$ -- mehr als zwei Klassen.
Jedes Dokument wird mindestens einer Klasse zugeordnet.
Tabelle \ref{tab:class_meaning} gibt einen Überblick über die Anteil und Bedeutung der im Datensatz verwendeten Klassenlabels.
%ist es nicht cooler die Häufigkeiten in % anzugeben? ich finds iwie übersichtlicher..



\begin{table}[h]
    \centering
    \begin{tabular}{l|l|l}
        \tiny\textbf{Klasse} & \tiny\textbf{Bedeutung} & \tiny\textbf{Anteil}\\
        \hline
        \tiny AC & \tiny Automation \& Control Systems                      &\tiny  4.76 \%  \\
        \tiny EV & \tiny Computer Science, Interdisciplinary Applications   &\tiny  8.61 \%  \\
        \tiny EW & \tiny Computer Science, Software Engineering             &\tiny  4.99 \%  \\
        \tiny EX & \tiny Computer Science, Theory \& Methods                &\tiny  9.63 \%  \\
        \tiny MC & \tiny Mathematical \& Computational Biology              &\tiny  2.93 \%  \\
        \tiny PE & \tiny Operations Research \& Management Science          &\tiny  6.03 \%  \\
        \tiny PN & \tiny Mathematics, Applied                               &\tiny  18.78 \% \\
        \tiny PO & \tiny Mathematics, Interdisciplinary Applications        &\tiny  5.81 \%  \\
        \tiny PQ & \tiny Mathematics                                        &\tiny  18.02 \% \\
        \tiny PS & \tiny Social Sciences, Mathematical Methods              &\tiny  1.81 \%  \\
        \tiny QL & \tiny LOGIC                                              &\tiny  0.10 \%  \\
        \tiny UR & \tiny Physics, Mathematical                              &\tiny  10.58 \% \\
        \tiny VS & \tiny Psychology, Mathematical                           &\tiny  0.60 \%  \\
        \tiny XY & \tiny Statistics \& Probability                          &\tiny  7.35 \%  \\
    \end{tabular}
    \caption{Bedeutung und Anteil der Klassen im Datensatz}
    \label{tab:class_meaning}
\end{table}


\section{Methoden}
\subsection{Sampling}

Zunächst haben wir den Datensatz in Test- und Trainingsdaten im Verhältnis 3:7 aufgeteilt, um die gelernten Modelle auf ungesehen Daten testen zu können.
Für die Aufteilung wurde \emph{stratified sampling} verwendet, sodass die Klassen anteilig in Test- und Trainingsdaten vertreten sind.
Wichtig hierbei ist es, dass wir anstatt der einzelnen Klassen die vergebenen Klassenkombinationen verwenden, sodass die Kombinationen von Klassen anteilig in den beiden Datensätzen auftreten.

\subsection{Multilabel Klassifikation}
\label{sub:multilabel_klassifikation}
Es existieren zwei Arten von Ansätzen, um das Multilabel Klassifikation Problem zu lösen.

Um mit gängigen Verfahren zur Klassifikation das Problem zu lösen werden die Labels transformiert,
so dass die Labels von bekannten Algorithmen verwendet werden können. Wir verwenden die Transformation der Labelkombinationen,
dass heißt jede Kombination der Labels wird als eine Klasse angesehen, wobei die Labels sortiert sind. Also werden die Vergaben der Labels
$l_1,l,2$ und $l_2,l_1$ als gleich angesehen.

Die andere Möglichkeit besteht darin bestehende Verfahren und Algorithmen auf das Mutlilabel Klassifikationproblem zu adaptieren.
Wir verwenden hierfür einen sogenannten One-Vs-Rest-Klassifikator, der für jede Klasse $c_i$ einen binären Klassifikator lernt.
Weitere Möglichkeiten um das Problem der Multilabelklassifikation anzugehen sind in \cite{Tsoumakas07multi-labelclassification:} zu finden.

\subsection{Support Vector Machines}
\label{sub:support_vector_machines}
Die \emph{Support Vector Machine} (SVM) ist ein Verfahren des überwachten Lernens, das zum Klassifizieren von Objekten verwendet wird.
Zum Lernen einer SVM werden die Trainingsdaten als Vektoren, zusammen mit den zugehörigen Klassen übergeben.
Für Texte ist die Repräsentation als Vektor meistens das Vektorspace-Modell, bei dem jedes Wort in dem Text als Feature angesehen wird.
Die SVM versucht dann anhand dieser Daten eine Hyperebene in den Raum zu legen, so dass die Trainingsdaten in zwei Klassen eingeteilt werden.
Der Abstand der Vektoren, die der Hyperebene am nächsten liegen, wird dabei maximiert.
Ursprünglich wurden SVMs zur Unterteilung der Objekte in nur zwei Klassen konzipiert.
Es wurde gezeigt, dass Support Vector Machines eines der besten Verfahren zur Klassifikation sind \cite{Joachims:1998:TCS:645326.649721}.

Um Support Vector Machines für das Multilabel Klassifizierungsproblem zu adaptieren, wird für jede Klasse $c \in L$ ein $\tilde c$ bestimmt, sodass für ein Dokument $D_i$ mit den Labels $L_i$ gilt:
\[
    \tilde c =
    \begin{cases}
        1 &\mbox{wenn } c \in L_i \\
        0 &\mbox{sonst}
    \end{cases}
\]

wobei $\tilde c$ die neue zu lernende Klasse für den binären Klassifkator repräsentiert.
Zur Klassifikation von neuen ungesehen Dokumenten wird nun jeder der binären Klassifikatoren gefragt, ob das aktuelle Dokument zu der gelernten Klasse gehört.
Ist die Antwort positiv, wird sich für dieses Dokument diese Klasse notiert, so dass man am Ende eine Menge von Klassen für das Dokument erhält.


\subsection{Latent Dirichlet Allocation}
\label{sub:latent_dirichlet_allocation}

Die \emph{Latent Dirichlet Allocation} (LDA) ist ein generatives stochastisches Modell und wurde in \cite{Blei:2003:LDA:944919.944937} eingeführt.
LDA lernt die in einer Textkollektion zugrundeliegenen Themen(Topics) der Dokumente. Die Anzahl der Topics muss vorher bekannt sein.
Die Topics sind Dirichletverteilungen über Wörtern und die Dokumente werden wiederum als eine Dirichletverteilung über Topics modelliert.
Über diese Verteilungen lassen sich Ähnlichkeiten der Dokumente ableiten, da sie ähnliche zugrundeliegende Themen diskutieren.

Die Verteilung über die Topics verwenden wir als neue Merkmale für die Dokumenten, um sie mit einer Support Vektor Machine zu klassifizieren.
Da dadurch die Anzahl der Features sehr viel kleiner wird bei uns max. 190 Merkmale, kann man von einer Art Featurereduktion sprechen.

\subsection{Evaluierungsmaße}
Um die erzielten Ergebnisse auswerten und mit einander vergleichen zu können, haben wir die gängigen Maße \emph{Precision}, \emph{Recall} und \emph{F1} genutzt.
Vollständigkeitshalber werden hier nochmal die entsprechenden Gleichungen aufgeführt.

Precision \[\frac{\#(\text{relevant items retrieved})}{\#(\text{retrieved items})}\]
Recall \[\frac{\#(\text{relevant items retrieved})}{\#(\text{relevant items})}\]
F1-Measure \[F_1 = 2 \cdot \frac{\mathrm{precision} \cdot \mathrm{recall}}{\mathrm{precision} + \mathrm{recall}}\]

Ferner wurden zwei verschiedene Ansätze genutzt, um die Ergebnisse für ein Verfahren über alle Klassen zusammenzufassen.
Es wurden über alle Klassen sowohl die Micro- als auch die Macro-Precision, Recall und F1-Measure berechnet.
Die Macro-Average-Maße berechnen einen einfachen Durchschnitt über alle Klassen, während die Micro-Average-Maße die Klassen nach deren Größen gewichten \cite{Manning:2008:IIR:1394399}.

% TODO: micro/macro average F1-Measure --> nachdem P. die Ergebnisse korrigiert hat
%"Macroaveraging computes a simple average over classes.....
%The differences between the two methods can be large. Macroaveraging
%gives equal weight to each class, whereas microaveraging gives equal weight
%to each per-document classification decision."

\section{Ergebnisse}

\subsection{Unigram SVM}
\label{sub:unigram_svm}

Zunächst haben wir die Support Vector Machine auf einem Unigram Vector Space Model gelernt,
um einen Vergleichswert für unseren Ansatz zu haben.
Dabei wurde jedes Dokument als Vektor modelliert, in dem der $i$-te Eintrag angibt, wie oft das $i$-te Wort aus dem Korpus in diesem Dokument vorkommt.

Für das Lernen einer Unigram SVM wurden zwei unterschiedliche Verfahren angewandt.
Beim ersten Verfahren wurde das Multilabel Klassifikationsproblem auf ein Multiclass Klassifikationsproblem reduziert,
indem jede Labelkombination als eigene Klasse angesehen wird.
Die Labelkombinationen wurden sortiert.
Das zweite Verfahren verwendet den One-Vs-Rest-Klassifizierer(OvR), der für jede Klasse einen binären Klassifikator lernt,
gegen alle anderen Datensätze, die diese Klasse nicht erhalten.

Für die SVMs wurde ein linearer Kernel verwendet und mit der Klasse LinearSVC aus dem \emph{python} Machine Learning Package \emph{scikit-learn} \cite{scikit-learn} trainiert.
Der Bestrafungsparameter $C = 0.01$ wurde durch mehrere Durchläufe mit $C \in \{ 10^{-4},10^{-3}, \dots, 10^4 \}$ ermittelt.

In Tabelle \ref{tab:unigram_svm} werden die verschiedene Metriken für beide der Unigram SVM Verfahren evaluiert.

\begin{table}[h]
    \centering
    \begin{tabular}{r|cc}
        \small \textbf{Metrik} & \small\textbf{Labelkombinationen} & \small\textbf{One-Vs-Rest}\\
        \hline
        \small \textbf{Macro Precision Score}  & \small 58.1 & \small  \textbf{58.4}\\
        \small \textbf{Macro Recall Score}     & \small \textbf{54.9} & \small 43.3\\
        \small \textbf{Macro F1 Score}        & \small \textbf{56.5} & \small 49.7\\
        \small \textbf{Micro Precision Score} & \small \textbf{62.2} & \small 56.2\\
        \small \textbf{Micro Recall Score}    & \small \textbf{61.5} & \small 53.8\\
        \small \textbf{Micro F1 Score}        & \small \textbf{61.8} & \small 55.0\\
    \end{tabular}
    \caption{Auswertung der Unigram SVM}
    \label{tab:unigram_svm}
\end{table}

Nach der Auswertung der Durschnitte über alle Klassen ist in Abbildung \ref{fig:svm_text_eval} das F1-Maß für jede der einzelnen
Klassen für beide Varianten aufgetragen.

\begin{figure}[H]
    \centering
    \def\svgwitdth{0.1\columnwidth}
    \input{figures/text_svm.pdf_tex}
    \caption{F1-Maß der Unigram SVM}
    \label{fig:svm_text_eval}
\end{figure}


\subsection{Topics mit LDA}
\label{sub:topics}
Eine Repräsentation der Textdokumente im Vektorspace-Modell liefert hoch-dimensionale Daten, wobei jedes Dokument mehr als $1$ Millionen Merkmale besitzt.
Dabei sind die Vektoren, die die einzelnen Dokumente darstellen, sehr dünn besetzt.
Das hat die Erprobung einer Dimensionsreduktion mithilfe von LDA motiviert.

Die Topics, die für die Featurereduktion verwendet werden sollen, wurden mit Mallet \cite{McCallumMALLET} gelernt.
Hierfür wurden die Stopwörter entfernt und wieder nur das Unigram Modell verwendet um die Topics zu lernen.
Es wurden mehrere Experimente durchgeführt, jeweils mit steigender Anzahl von Topics: $10$, $40$, $70$, $100$, $130$, $160$, $190$.
In Tabelle \ref{tab:topics_words} befinden sich einige der gelernten Topics der Trainingsdokumente aus dem Topicmodell mit 100 Topics.
Dargestellt werden die Topics durch die Wörter mit der höchsten Wahrscheinlichkeit.

\begin{table}[h]
    \centering
    \begin{tabular}{c|l}
        \small \textbf{Topic Nummer} & \small\textbf{Wörter}\\
        \hline
        \small \textbf{01} & \small system systems paper dynamic presented developed distributed analysis\\
        \small \textbf{10} & \small inequalities inequality convex variational functions monotone principle\\
        \small \textbf{20} & \small bound bounds lower upper number show results case tight give bounded obtain \\
        \small \textbf{24} & \small alpha bar gamma phi beta vertical mu theta element pi infinity \\
        \small \textbf{30} & \small noise signal signals source channel sources noisy frequency \\
        \small \textbf{40} & \small molecular energy calculations molecules bond density electron \\
        \small \textbf{50} & \small dimension fractal brownian motion box super law fractional \\
        \small \textbf{60} & \small bifurcation center dot point points critical hopf parameter \\
        \small \textbf{70} & \small approach range applications wide methodology results \\
        \small \textbf{80} & \small class classes properties paper general show property special \\
        \small \textbf{90} & \small flow flows velocity stokes fluid navier turbulent vortex \\
        \small \textbf{100}& \small case cases general special results ii study paper iii show
    \end{tabular}
    \caption{Auszug der wahrscheinlichsten Wörter einiger Topics}
    \label{tab:topics_words}
\end{table}

Nach dem Trainieren der Topics auf dem Trainingsdatensatz, wurde dieses Modell dann auf die ungesehenen Dokumente aus
dem Testdatensatz angewandt.
Dadurch erhalten die Testdokumente ebenfalls eine Verteilung über die gelernten Topics.

\subsection{Topic SVM}
\label{sub:topic_svm}
Training- und Testdatensatz werden jetzt durch die Verteilung über ihre Topics dargestellt.
Bei einem Topicmodel mit $n$ Topics
werden die Dokumente nur noch durch $n$ numerische Features repräsentiert und es gilt $\forall p_i: 0 \le p_i \le 1 $, wobei $p_i$
die Wahrscheinlichkeit für das i-te Topic ist.

Wie bei der Unigram SVM werden wieder die beiden Varianten Labelkombinationen und One-Vs-Rest ausgewertet.
Das Modell für die SVM wurde auf den Trainingsdokumenten als Topicverteilung gelernt und dann auf die Testdokumente angewandt.

In den Tabellen \ref{tab:topics_svm_labelcombs} und \ref{tab:topics_svm_ovr} werden die Ergebnisse anhand der Anzahl der Topics
dargestellt.

\begin{table}[h]
    \begin{tabular}{r|ccccccc}
        \tiny\textbf{Metrik} & \tiny\textbf{10 Topics} &\tiny \textbf{40 Topics} &\tiny \textbf{70 Topics} &\tiny \textbf{100 Topics} & \tiny \textbf{130 Topics} &  \tiny \textbf{160 Topics} &  \tiny \textbf{190 Topics} \\
        \hline
        \tiny \textbf{Macro Precision Score}  & \tiny 9.19 & \tiny 8.90& \tiny \textbf{9.88}&\tiny 8.47&\tiny 9.58&\tiny 9.20&\tiny 9.19\\
        \tiny \textbf{Macro Recall Score}     & \tiny 8.20 & \tiny 8.35& \tiny \textbf{8.44}&\tiny 7.89&\tiny 8.30&\tiny 8.05&\tiny 7.96\\
        \tiny \textbf{Macro F1 Score}        & \tiny 8.67 & \tiny 8.62& \tiny \textbf{9.11}&\tiny 8.17&\tiny 8.89&\tiny 8.59&\tiny 8.53\\
        \tiny \textbf{Micro Precision Score} & \tiny 11.3 & \tiny 11.3& \tiny 11.5&\tiny \textbf{18.8}&\tiny 11.4&\tiny 10.7&\tiny 10.9\\
        \tiny \textbf{Micro Recall Score}    & \tiny 8.88 & \tiny 8.66& \tiny 9.07&\tiny \textbf{14.8}&\tiny 8.98&\tiny 8.47&\tiny 8.58\\
        \tiny \textbf{Micro F1 Score}        & \tiny 9.95 & \tiny 9.70& \tiny 10.1&\tiny \textbf{16.6}&\tiny 10.0&\tiny 9.49&\tiny 9.61\\
    \end{tabular}
    \caption{Ergebnisse der SVM mit Topics und Labelkombinationen}
    \label{tab:topics_svm_labelcombs}
\end{table}

\begin{table}[h]
    \begin{tabular}{r|ccccccc}
        \tiny\textbf{Metrik} & \tiny\textbf{10 Topics} &\tiny \textbf{40 Topics} &\tiny \textbf{70 Topics} &\tiny \textbf{100 Topics} & \tiny \textbf{130 Topics} &  \tiny \textbf{160 Topics} &  \tiny \textbf{190 Topics} \\
        \hline
        \tiny \textbf{Macro Precision Score}  & \tiny 7.26& \tiny 9.31& \tiny \textbf{10.5}&\tiny 7.31&\tiny 8.51&\tiny 10.4&\tiny 8.72\\
        \tiny \textbf{Macro Recall Score}     & \tiny 0.01& \tiny 0.06& \tiny 0.06&\tiny \textbf{3.89}&\tiny 0.07&\tiny 0.05&\tiny 0.06\\
        \tiny \textbf{Macro F1 Score}        & \tiny 0.02& \tiny 0.12& \tiny 0.12&\tiny \textbf{5.08}&\tiny 0.14&\tiny 0.10&\tiny 0.12\\
        \tiny \textbf{Micro Precision Score} & \tiny 0.01& \tiny 0.08& \tiny 0.08&\tiny \textbf{11.7}&\tiny 0.08&\tiny 0.06&\tiny 0.07\\
        \tiny \textbf{Micro Recall Score}    & \tiny 0.01& \tiny 0.06& \tiny 0.06&\tiny \textbf{9.82}&\tiny 0.06&\tiny 0.05&\tiny 0.06\\
        \tiny \textbf{Micro F1 Score}        & \tiny 0.01& \tiny 0.07& \tiny 0.07&\tiny \textbf{10.6}&\tiny 0.07&\tiny 0.05&\tiny 0.06\\
    \end{tabular}
    \caption{Ergebnisse der SVM mit Topics und One-Vs-Rest}
    \label{tab:topics_svm_ovr}
\end{table}

Wie für die Unigram SVM sind in Abbildung \ref{fig:svm_topic_eval} die F1-Maße für die Klassen aufgetragen,
diesmal jedoch in zwei unterschiedlichen Graphen, um die Werte der verschiedenen Topicgrößen zu einem der beiden Verfahren miteinander
zu vergleichen.

% ?? also, das eine ist Labelkombinationen, das andere OvR oder wie?

\begin{figure}[H]
    \centering
    \def\svgwitdth{0.1\columnwidth}
    \input{figures/eval_svm_labelwise_plot.pdf_tex}
    \caption{Ergebnisse der Klassifizierung mit Topics als Vorverarbeitung: (oben) Labelkombinationen (unten) One-Vs-Rest}
    \label{fig:svm_topic_eval}
\end{figure}

\section{Diskussion}

\subsection{Unigram SVM}
Die Klassifizierung des Testdatensatzes mit einer Unigram SVM liefert eine F1-Score von $50$ bis $60 \%$.
Obwohl die Labelkombination und die One-vs-Rest SVM fast die gleiche Macro Precision haben, zeigen die restlichen Metriken deutlich, dass die Labelkombinationen SVM um etwa $10 \%$ bessere Werte liefert als die One-vs-Rest SVM.


\subsection{Unigram SVM mit Topicmodells}
Es ist festzuhalten, dass je mehr Topics das zu lernende Topicmodell hat, desto resourcen-aufwendiger werden die Berechnungen.
Aufgrund vom wachsenden Verbrauch von Speicher und Arbeitsspeicher war es nicht möglich Modelle mit mehr als $190$ Topics zu lernen.

Es wird beobachtet, dass die SVM mit Topics und Labelkombinationen deutlich bessere Ergebnisse liefert als die eine mit One-vs-Rest Klassifizierer.
Bei der Klassifizierung mit Labelkombinationen gibt es keine gewaltigen Unterschiede bei den verschiedenen Topicmodells.
Wir bemerken jedoch, dass bei einer nach der Klassengrößen gewichteten Bildung der Durchschnittswerte sich das Modell mit $100$ Topics abhebt.
Bei der Klassifizierung mit One-vs-Rest Klassifizierer schneidet das Modell mit $100$ Topics deutlich besser ab als alle anderen Modelle.
Es lässt sich auch keine Tendenz besserer Resultate mit steigender Anzahl der gelernten Topics feststellen.


\subsection{Vergleich beider Verfahren}
Wenn wir die Ergebnisse beider Ansätze mit einander vergleichen, erkennen wir, dass die Unigram SVMs um mehr als das fünf-fache besser abschneiden, als die SVM auf dem besten Topicmodell
(je nach Gewichtung, das Modell mit $70$ oder mit $100$ Topics) -- bei der Unigram Labelkombinationen SVM haben wir eine Micro F1-Score von $61.8 \%$, während die Micro F1-Score von der SVM mit Topicmodells und Labelkombinationen nur $16.6 \%$ beträgt.
In einem ähnlichen Verhältnis zu einander stehen auch alle restlichen Metriken.
Also wurde die Annahme, dass eine Dimensionsreduktion mit LDA eine bessere Multilabel-Klassifikation ermöglichen wird, widerlegt.



% Probleme
% Speicherplatz (zb mit steigender Anzahl von Topics)

\section{Zusammenfassung}

In dieser Arbeit haben wir zwei Ansätze zur Multilabel-Klassifikation von wissenschaftlichen Publikationen anhand von deren Titeln und Abstracts implementiert und verglichen.
Zunächst wurden die Publikationen direkt mithilfe einer Unigram SVM klassifiziert.
Hier haben wir sowohl eine Labelkombinationen als auch eine One-vs-Rest SVM gelernt und festgestellt, dass die Labelkombinationen SVM, mit einer Micro F1-Score von $61.8 \%$ auf dem Testdaten, bessere Ergebnisse liefert.
Später wurde der Zwischenschritt einer Dimensionsreduktion mit LDA implementiert.
Obwohl es mehrere Modelle mit unterschiedlicher Topicanzahl gelernt wurden, waren die Ergebnisse der Klassifizierung bei diesem Ansatz deutlich schlechter als bei der Unigram SVM.
Die Hypothese, dass eine Dimensionsreduktion der Daten mittels LDA Topicmodells das Klassifikationsproblem optimieren wird, hat sich nicht bestätigt.

% was ist hierbei rausgekommen?
% welches war besser?
% was war unsere vermutung, hat sie sich bestätigt?
% ausblick: wofür ist das gut? kann man das iwie weiter verbessern? oder vielleicht in eine andere richtung gehen?


%\section{Ergebnisse}
%\section{Diskussion}
%\section{Zusammenfassung}

\nocite{*}
%\newpage
\bibliography{literature}
\bibliographystyle{alpha}

\end{document}
