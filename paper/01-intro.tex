\section{Einleitung}

\begin{comment}
Einführung und Einbettung des Problems
Was ist die konkrete Problemstellung / Fragestellung der Arbeit
Literature Review
\end{comment}

Die Klassifikation von Texten ist in den letzten Jahren, mit der steigenden Anzahl an Informationen und Entstehung von Textsammlungen, zu einem wichtigen Werkzeug für Organisation und Handhabung geworden.
Viele praktische Anwendungen nutzen Textklassifikation als Verfahren, um die Benutzung von Texten zu vereinfachen und zugänglicher zu machen.
Beispiele dafür sind Bibliothekssysteme, Empfehlungssysteme, Spamerkennung, Erkennung der Sprache in Texten oder der Themen in Nachrichten.

Traditionell wird bei Textklassifikation jedem Dokument $d$ exakt ein Label $l$ aus einer Menge von Labels $L$ zugewiesen.
Wenn $|L| = 2$, dann spricht man von einem binären Klassifikationsproblem, wenn $|L| > 2$ -- von einem Multiclass-Problem.
Bei Multilabel-Textklassifikation werden Dokumenten eine Menge von Labels $Y \in L$ zugeordnet.
Die Multiclass-Klassifikation hat auch ihren Ursprung in der Textklassifikation, da dort häufig die Anforderung besteht, Dokumente in verschiedene Gruppen (mehr als 2) einzuteilen (ein Beispiel hierfür wäre die Einsortierung von Nachrichten in verschiedene Kategorien).
Das Gleiche gilt auch für die Multilabel-Klassifikation, da es oftmals sinnvoll ist, ein Dokument gleichzeitig mehreren Kategorien zuzuordnen (die selbe Nachricht könnte z.B. gleichzeitig die Labels ``Politik'', ``Wirtschaft'' und ``International'' erhalten).

In dieser Arbeit betrachten wir die Auswirkungen von der \emph{Latent Dirichlet Allocation} (LDA) zur Vorverarbeitung für die Multilabel-Klassifikation mit linearen \emph{Support Vector Machines} (SVM).
Es wird überprüft, inwiefern eine Dimensionsreduktion des Featureraums mit Hilfe der LDA eine optimalere Lösung des Multilabel-Klassifikation-Problems liefert als die direkte Anwendung linearer SVM auf den selben Daten.

\subsection{Datensatz}
\label{sub:datensatz}
%woher kommt der Datensatz?
Der untersuchte Datensatz besteht aus $1.1$ Millionen mathematischen Publikationen mit Titel, Abstract, Klassen und Erscheinungsjahr.
Der schematische Aufbau des Datensatzes wird in Tabelle \ref{tab:data} dargestellt.

%[Doc:] [ID] [LABELS] [YEAR] [TITLE] [ABSTRACT]
\begin{table}[h]
    %\rowcolors[]{1}{blue!20}{blue!10}
    \begin{tabular}{cccll}
        \tiny\textbf{ID} &\tiny \textbf{CLASSES} &\tiny \textbf{YEAR} &\tiny \textbf{TITLE} & \tiny \textbf{ABSTRACT} \\
        \hline
        \tiny 1000000796 &\tiny EW & \tiny 2000 & \tiny Dynamo: A transparent dynamic \dots  & \tiny We describe the design \dots \\
        \tiny 1000000815 &\tiny EW & \tiny 2000 & \tiny Unification-based pointer \dots  & \tiny This paper describes \dots \\
        \multicolumn{5}{c}{\dots} \\
        \tiny 1000003814 &\tiny EV,PE & \tiny 1995 & \tiny A GENETIC APPROACH TO \dots  & \tiny The quadratic assignment \dots \\
        \tiny 1000004333 & \tiny PQ,EX &\tiny 1995 &\tiny DYNAMIC EUCLIDEAN MINIMUM \dots & \tiny We maintain the minimum \dots \\
    \end{tabular}
    \caption{Auszug aus dem Datensatz}
    \label{tab:data}
\end{table}


Die Publikationen werden in 14 verschiedene Klassen eingeteilt.
Für $75.6 \%$ der Dokumente wurde nur eine Klasse vergeben, für $24.3 \%$ -- zwei Klassen und für $0.1 \%$ -- mehr als zwei Klassen.
Jedes Dokument wird mindestens einer Klasse zugeordnet.
Tabelle \ref{tab:class_meaning} gibt einen Überblick über die Anteil und Bedeutung der im Datensatz verwendeten Klassenlabels.
%ist es nicht cooler die Häufigkeiten in % anzugeben? ich finds iwie übersichtlicher..



\begin{table}[h]
    \centering
    \begin{tabular}{l|l|l}
        \tiny\textbf{Klasse} & \tiny\textbf{Bedeutung} & \tiny\textbf{Anteil}\\
        \hline
        \tiny AC & \tiny Automation \& Control Systems                      &\tiny  4.76 \%  \\
        \tiny EV & \tiny Computer Science, Interdisciplinary Applications   &\tiny  8.61 \%  \\
        \tiny EW & \tiny Computer Science, Software Engineering             &\tiny  4.99 \%  \\
        \tiny EX & \tiny Computer Science, Theory \& Methods                &\tiny  9.63 \%  \\
        \tiny MC & \tiny Mathematical \& Computational Biology              &\tiny  2.93 \%  \\
        \tiny PE & \tiny Operations Research \& Management Science          &\tiny  6.03 \%  \\
        \tiny PN & \tiny Mathematics, Applied                               &\tiny  18.78 \% \\
        \tiny PO & \tiny Mathematics, Interdisciplinary Applications        &\tiny  5.81 \%  \\
        \tiny PQ & \tiny Mathematics                                        &\tiny  18.02 \% \\
        \tiny PS & \tiny Social Sciences, Mathematical Methods              &\tiny  1.81 \%  \\
        \tiny QL & \tiny LOGIC                                              &\tiny  0.10 \%  \\
        \tiny UR & \tiny Physics, Mathematical                              &\tiny  10.58 \% \\
        \tiny VS & \tiny Psychology, Mathematical                           &\tiny  0.60 \%  \\
        \tiny XY & \tiny Statistics \& Probability                          &\tiny  7.35 \%  \\
    \end{tabular}
    \caption{Bedeutung und Anteil der Klassen im Datensatz}
    \label{tab:class_meaning}
\end{table}

