\section{Einleitung}

Die Klassifikation von fexten ist in den letzt,n Ja:ren mit der steigenden Anzahl an Informationen und Entstehung von Textsammlungen zur einem wichtigen Verfahren zur Organisation und Handhabung geworden. Viele praktische Anwendungen nutzen Textklassifikation als Verfahren um Texte für Benutzer besser zu Nutzbar zu machen, so zum Beispiel Bibliothekssysteme, Empfehlungssysteme, Spamerkennung, Erkennung der Sprache in Texten oder der Erkennung von Themen in Nachrichten.

Traditionell wurde jedem Dokument exakt ein Label $l$ aus einer Menge von Labels $L$ zugewiesen. Wenn $|L| = 2$, dann spricht man von einem binären Klassifiaktionsproblem und wenn $|L| > 2$ von einem Multiclass Problem. Bei der Multilabeltextklassifikation werden Dokumenten eine Menge von Labels $Y \in L$ zugeordnet. Die Multilabelklassifikation hat auch ihren Ursprung in der Textklassifikation, da dort häufig die Anforderung besteht Dokumenten in verschiedene Gruppen einzuteilen.

In dieser Arbeit werden wir die Auswirkungen von der Latent Dirichlet Allocation zur Vorverarbeitung für die Multilabelklassifikation mit Support Vector Machines betrachten.

\subsection{Datensatz}
Der Datensatz besteht aus $1.1$ Millionen mathematischen Publikationen mit Title, Abstract, Klassen und Erscheinungsjahr. Die Publikationen werden in 14 verschiedene Klassen eingeteilt. Für $75.6 \%$ der Dokumente wurde nur eine Klasse vergeben, für $24.3 \%$ zwei Klassen und $0.1 \%$ mehr als zwei Klassen. Jedes Dokument wird mindestens einer Klasse zugeordnet.
\label{sub:datensatz}
\begin{table}[h]
    \centering
    \begin{tabular}{l|l|l}
        \textbf{Klasse} & \textbf{Bedeutung} & \textbf{Häufigkeit}\\
        \hline
        AC & Automation \& Control Systems & 67450\\
        EV & Computer Science, Interdisciplinary Applications & 122025\\
        EW & Computer Science, Software Engineering & 70796\\
        EX & Computer Science, Theory \& Methods & 136456\\
        MC & Mathematical \& Computational Biology & 41590\\
        PE & Operations Research \& Management Science & 85476\\
        PN & Mathematics, Applied & 266198\\
        PO & Mathematics, Interdisciplinary Applications & 82404\\
        PQ & Mathematics & 255421\\
        PS & Social Sciences, Mathematical Methods & 25679\\
        QL & LOGIC & 1448\\
        UR & Physics, Mathematical & 149917\\
        VS & Psychology, Mathematical & 8439\\
        XY & Statistics \& Probability & 104227\\
    \end{tabular}
    \caption{Bedeutung und Häufigkeit der Klassen im Datensatz}
\end{table}


\subsection{Multilabel Klassifikation}
\label{sub:multilabel_klassifikation}
Bei der Multilabel Klassifikation existieren zwei Arten von Ansätzen, um das Problem zu lösen. Bei der ersten wird das Problem transformiert und bei der zweiten werden existierende Verfahren adaptiert.

\subsection{Support Vector Machines}
\label{sub:support_vector_machines}
Die  Support Vector Machine(SVM) ein Verfahren des unüberwachten Lernen und kann zum Klassifizieren von Objekten verwendet werden. Zum Lernen einer SVM werden die Trainingsdaten als Vektoren mit den zugehörigen Klassen übergeben. Für Texte ist die Repräsentation als Vektor meistens das Vektorspace-Modell, bie dem jedes Wort in dem Text als Feature angesehen wird. Die SVM versucht dann anhand dieser Daten eine Hyperebene so in den Raum zu legen, so dass die Trainingsdaten in zwei Klassen eingeteilt werden. Der Abstand der Vektoren die der Hyperebene am nächsten liegen wird dabei maximiert. Ursprünglich wurden SVMs zur Unterteilung der Objekte in nur zwei Klassen konzipiert.
Es wurde gezeigt, dass Support Vector Machines eines der besten Verfahren zur Klassifikation ist \cite{Joachims:1998:TCS:645326.649721}.

%\begin{figure}[h]
    %\centering
    %\def\svgwitdth{0.75\columnwidth}
    %\input{figures/svm_intro.pdf_tex}
    %\caption{Hypereben einer SVM im 2-dimensionalen}
    %\label{fig:svm_intro}
    %\footnote*{http://commons.wikimedia.org/wiki/File:Svm\_intro.svg}
%\end{figure}


%Die Support Vector Machine(SVM) ist ein weit verbreitetes Verfahren zur Klassifizierung im Bereich des maschinellen Lernens. Es handelt sich dabei um ein Ve

% unüberwachtes Lernen



Um Support Vector Machines auf das Multilabel Klassifizierungsproblem zu adaptieren wird für jede Klasse $c \in L$, sodass für ein Dokument $D_i$ mit den Labels $L_i$ gilt:
\[
    \tilde c =
    \begin{cases}
        1 &\mbox{wenn } c \in L_i \\
        0 &\mbox{sonst}
    \end{cases}
\]

Wobei $\tilde c$ die neue zu lernende Klasse für den  binären Klassifkator repräsentiert.
Zur Klassifikation von neuen ungesehen Dokumenten wird nun jeder der binären Klassifikatoren gefragt, ob das aktuelle Dokument zu der gelernten Klasse gehört. Ist die Antwort positiv wird sich für dieses Dokument diese Klasse notiert, so dass man am Ende eine Menge von Klassen für dieses Dokument erhält.




\subsection{Latent Dirichlet Allocation}
\label{sub:latent_dirichlet_allocation}
