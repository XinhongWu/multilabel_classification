\section{Zusammenfassung}

In dieser Arbeit haben wir zwei Ansätze zur Multilabel-Klassifikation von wissenschaftlichen Publikationen anhand von deren Titeln und Abstracts implementiert und verglichen.
Zunächst wurden die Publikationen direkt mit hilfe einer Unigram SVM klassifiziert.
Hier haben wir folgende Erkenntnisse gewonnen/folgende Ergebnisse erzielt....
Später wurde der Zwischenschritt einer Dimensionsreduktion mit LDA implementiert.
Obwohl es mehrere Modelle mit unterschiedlicher Topicanzahl gelernt wurden, waren die Klassifizierungsergebnisse bei diesem Ansatz deutlich schlechter als bei der Unigram SVM.
Die Hypothese, dass eine Dimensionsreduktion der Daten mittels LDA Topicmodells das Klassifikationsproblem optimieren wird, hat sich nicht bestätigt.

% was ist hierbei rausgekommen?
% welches war besser?
% was war unsere vermutung, hat sie sich bestätigt?
% ausblick: wofür ist das gut? kann man das iwie weiter verbessern? oder vielleicht in eine andere richtung gehen?
