\section{Methoden}

%Datensatz
%Verwendete Tools
%Eigene Implementierung

\subsection{Datensatz}
Der Datensatz besteht aus $1.1$ Millionen mathematischen Publikationen mit Title, Abstract, Klassen und Erscheinungsjahr. Die Publikationen werden in 14 verschiedene Klassen eingeteilt. Für $75.6 \%$ der Dokumente wurde nur eine Klasse vergeben, für $24.3 \%$ zwei Klassen und $0.1 \%$ mehr als zwei Klassen. Jedes Dokument wird mindestens einer Klasse zugeordnet.
\label{sub:datensatz}
\begin{table}[h]
    \centering
    \begin{tabular}{l|l|l}
        \textbf{Klasse} & \textbf{Bedeutung} & \textbf{Häufigkeit}\\
        \hline
        AC & Automation \& Control Systems & 67450\\
        EV & Computer Science, Interdisciplinary Applications & 122025\\
        EW & Computer Science, Software Engineering & 70796\\
        EX & Computer Science, Theory \& Methods & 136456\\
        MC & Mathematical \& Computational Biology & 41590\\
        PE & Operations Research \& Management Science & 85476\\
        PN & Mathematics, Applied & 266198\\
        PO & Mathematics, Interdisciplinary Applications & 82404\\
        PQ & Mathematics & 255421\\
        PS & Social Sciences, Mathematical Methods & 25679\\
        QL & LOGIC & 1448\\
        UR & Physics, Mathematical & 149917\\
        VS & Psychology, Mathematical & 8439\\
        XY & Statistics \& Probability & 104227\\
    \end{tabular}
    \caption{Bedeutung und Häufigkeit der Klassen im Datensatz}
\end{table}



\subsection{Test- und Trainingsdaten}
\label{sub:test_und_trainingsdaten}
Zunächst haben wir unseren Datensatz in Test- und Trainingsdaten im Verhältnis 3:7 aufgeteilt, um unser gelerntes Modell auf ungesehen Daten zu testen. für die Aufteilung wurde das Stratified Sampling verwendet, sodass die Klassen anteilig in Test- und Trainingsdaten vertreten sind. Wir verwenden anstatt der einzelnen Klassen, die vergebenen Klassenkombinationen, sodass die Kombinationen von Klassen anteilig in den beiden Datensätzen auftreten.

% subsection test_und_trainingsdaten (end)
\subsection{Mutlilabel-Klassifikation}
% scikit-learn: LinearSVC

\subsection{Dimensionsreduktion mit LDA und Multilabel-Klassifikation}
% mallet für LDA
% scikit-learn: LinearSVC für die Klassifikation
