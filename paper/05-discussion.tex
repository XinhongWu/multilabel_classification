\section{Diskussion}

\subsection{Unigram SVM}
% Komentar zu den Unigram SVMs

\subsection{Unigram SVM mit Topicmodells}
% Komentar zu den Topicmodells
Es wird beobachtet, dass die SVM mit Topics und Labelkombinationen deutlich bessere Ergebnisse liefert als die eine mit One-vs-Rest Klassifizierer.
Bei der Klassifizierung mit Labelkombinationen gibt es keine gewaltigen Unterschiede bei den verschiedenen Topicmodells.
Jedoch bemerken wir, dass bei einer nach der Klassengrößen gewichteten Bildung der Durchschnittswerte sich das Modell mit 100 Topics abhebt.
% One-vs-Rest
% vgl zwischen beiden

\subsection{Vergleich zwischen den beiden Verfahren}
% Vergleich zwischen beiden Ansätzen
Wenn wir die Ergebnisse beider Ansätze mit einander vergleichen, lässt sich feststellen, dass die Unigram SVMs um mehr als das fünf-fache besser abschneiden, als die SVM auf dem besten Topicmodell
(je nach Gewichtung, das Modell mit 70 oder mit 100 Topics).


% Probleme
% Speicherplatz (zb mit steigender Anzahl von Topics)
